\documentclass[a4paper]{article}
\usepackage{fullpage}
\usepackage{amsmath}
\usepackage{amssymb}
\usepackage{bussproofs}

\newtheorem{lemma}{Lemma}
\newtheorem{definition}{Definition}

\newcommand{\MODEL}{\mathcal{M}}
\newcommand{\LANGUAGE}{\mathcal{L}}
\newcommand{\TUPLE}[1]{\langle {#1} \rangle}
\newcommand{\SET}[1]{\{ {#1} \}}
\newcommand{\PV}{\varphi}
\newcommand{\QV}{\psi}

\usepackage[
top=0.3in,
nohead]{geometry}

\title{PHIL3110 - Assignment 1}
\author{Maxwell Bo}

\begin{document} 

\maketitle

\section*{Part A}

\subsection*{Problem 1}

\begin{enumerate}
    \item 
    \begin{prooftree}
        \AxiomC{$Pa \vee Qa^{(1)}$}
        \AxiomC{$(Pa \vee Qa) \to \bot$}
        \BinaryInfC{$\bot$}
        \RightLabel{\scriptsize(1) $(\neg I)$}
        \UnaryInfC{$\neg(Pa \vee Qa)$}
        \RightLabel{\scriptsize(DM-1)}
        \UnaryInfC{$\neg Pa \wedge \neg Qa$}
    \end{prooftree}


    De Morgan's Law 1 

    \begin{prooftree}
                    \AxiomC{$\PV^{(1)}$}
                    \UnaryInfC{$\PV \vee \QV$}
                    \AxiomC{$\neg(\PV \vee \QV)$}
                \BinaryInfC{$\bot$}
                \RightLabel{\scriptsize(1) $(\neg I)$}
            \UnaryInfC{$\neg \PV$}
                    \AxiomC{$\QV^{(2)}$}
                    \UnaryInfC{$\PV \vee \QV$}
                    \AxiomC{$\neg(\PV \vee \QV)$}
                \BinaryInfC{$\bot$}
                \RightLabel{\scriptsize(2) $(\neg I)$}
            \UnaryInfC{$\neg \QV$}
        \BinaryInfC{$\neg \PV \wedge \neg \QV$}
    \end{prooftree}

    \item 
    \begin{prooftree}
        \AxiomC{$Qa \vee Ra^{(3)}$}
        \AxiomC{$Qa^{(1)}$}
        \AxiomC{$Qa \to Pa$}
        \BinaryInfC{$Pa$}
        \AxiomC{$Ra^{(2)}$}
        \AxiomC{$Ra \to Pa$}
        \BinaryInfC{$Pa$}
        \RightLabel{\scriptsize(1) \scriptsize(2) $(\vee E)$}
        \TrinaryInfC{$Pa$}
        \RightLabel{\scriptsize(3) $(\to I)$}
        \UnaryInfC{$(Qa \vee Ra) \to Pa$}
        % \RightLabel{\scriptsize(1) $\neg I$}
        % \UnaryInfC{$\neg(Pa \vee Qa)$}
    \end{prooftree}

    \item 
    \begin{prooftree}
        \AxiomC{$\neg (\forall x Px \to \exists x Rx)$}
        \AxiomC{$\forall x Px^{(1)}$}
        \AxiomC{$\exists x Rx^{(2)}$}
        \BinaryInfC{$\forall x Px \wedge \exists x Rx$}
        \UnaryInfC{$\exists x Rx$}
        \RightLabel{\scriptsize(1) $(\to I)$}
        \UnaryInfC{$\forall x Px \to \exists x Rx$}
        \BinaryInfC{$\bot$}
        \RightLabel{\scriptsize(2) $(\neg I)$}
        \UnaryInfC{$\neg \exists x Rx$}
    \end{prooftree}
\end{enumerate}

\subsection*{Problem 2}

\begin{enumerate}
    % TODO FIX UP THESE BRACKETS
    \item \begin{align*}
        Q^{\MODEL} &= \{ m_1 \}\\
        T^{\MODEL} &= \{ \TUPLE{m_1, m_1}, \TUPLE{m_1, m_2}, \TUPLE{m_2, m_2} \}
    \end{align*}

    \item Distressingly, $\LANGUAGE$ does not define any constant symbols, nor does $\MODEL$ provide interpretations of constant symbols in  $\MODEL$.

    Thus

    \[\mathcal{M} \nvDash \exists x\, \neg Txx\]

    However assuming $\MODEL^+$, where $\MODEL^+$ is the expanded model $\mathcal{M}$, where $m^{\MODEL} = m$ for all $m \in M$, we see that

    \[M \models \exists x\, \neg Txx\]

    as

    \[\TUPLE{m_{3}^{\MODEL}, m_{3}^{\MODEL}} \not\in T^{\MODEL}\]

    \item No, as $\MODEL$  does not define any constant symbols $\MODEL \nvDash \exists x\, \varphi$ for some arbitrary $\varphi$ (as $x$ will bind no constant symbols), and thus

    \[\MODEL \nvDash \exists x \forall y\, (Qy \leftrightarrow Tyx)\]

    Assuming $\MODEL^+$,


    \[\MODEL^+ \nvDash \exists x \forall y\, (Qy \leftrightarrow Tyx)\]
    
    By fixing $x$ to $m_{1}^{\MODEL}$, we see that

    \[\forall y \cdot y \in Q^{\MODEL} \leftrightarrow \TUPLE{y, m_{1}^\MODEL} \in T^{\MODEL}\]

    as 
    
    \begin{align*}
    m_{1}^\MODEL & \in Q^{\MODEL}\  \text{and}\ \TUPLE{m_{1}^\MODEL, m_{1}^\MODEL} \in T^{\MODEL}\\
    m_{2}^\MODEL & \not\in Q^{\MODEL}\  \text{and}\ \TUPLE{m_{2}^\MODEL, m_{1}^\MODEL} \not\in T^{\MODEL}\\
    m_{3}^\MODEL & \not\in Q^{\MODEL}\  \text{and}\ \TUPLE{m_{3}^\MODEL, m_{1}^\MODEL} \not\in T^{\MODEL}
    \end{align*}

\end{enumerate}

\section*{Part B}

\subsection*{Problem 3}

\begin{enumerate}
    \item \begin{enumerate}
        \item $BHMB$ is not a good sandwich
        \item $(((B)JB)HB)$ is a good sandwich
        \item $BHBMBJM$ is not a good sandwich
    
    \end{enumerate}

    \item \begin{definition}
    A sandwich is good iff there is some stage $n$ such that $sandwich \in Stage(n)$ where:

    \begin{itemize}
        \item $Stage(0) = \{ B \}$
        \item $Stage(n + 1)$ is the set of $\varphi$ such that either:
        \begin{enumerate}
            \item $\varphi \in Stage(n)$
            \item $\varphi$ is of the form $\psi MB$, $\psi HB$, or $\psi JB$, where $\psi \in Stage(n)$
        \end{enumerate}
    \end{itemize}

    \end{definition}

    \item Make $A$ the set of all $n$ such that for each $\psi \in Stage(n)$, $\psi$ does not contain any two instances of the same ingredient adjacently.

    \subsubsection*{Base} $0 \in A$. As $Stage(0) = \SET{B}$, all $\varphi \in Stage(0)$ do not contain any two instances of the same ingredient adjacently. We can also see that all $\varphi \in Stage(0)$ are terminated by $B$

    \subsubsection*{Induction Step}
    Suppose $n \in A$.
    
    \begin{lemma}
    Since $n \in A$, and every $\varphi \in Stage(n)$ is of the form $B$, $\psi MB$, $\psi HB$, or $\psi JB$, for $\psi \in Stage(m)$ for some $m < n$, all $\varphi$ are terminated by $B$.
    \end{lemma}

    \begin{lemma}
    Since $n \in A$, every $\varphi \in Stage(n)$ does not contain any two instances of the same ingredient adjacently.
    \end{lemma}

        For each $\varphi \in Stage(n + 1) \setminus Stage(n)$, recalling the definition for $Stage(n + 1)$, $\varphi$ is of the form $\psi MB$, $\psi HB$, or $\psi JB$, for some $\psi \in Stage(n)$

        Given \begin{itemize}
            \item all $\varphi \in Stage(n)$ are terminated by $B$ (Lemma 1)
            \item all $\varphi \in \SET{MB, HB, JB}$ don't begin with $B$, and are not themselves adjacent ingredients
            \item Lemma 2
        \end{itemize}
        
        all $\varphi \in Stage(n + 1) \setminus Stage(n)$ do not contain two instances of the same ingredient adjacently. 

        Thus, $n + 1 \in A$. Then by induction, we see that every $n \in A$.
    
\end{enumerate}

\subsection*{Problem 4}

\begin{enumerate}
    \item Suppose $d_{\psi}$ is a derivation of $\Gamma, \varphi \vdash \psi$, with $d_{\psi}$ being from stage $n$.

    The $(\to I)$ rule tells us that the $n + 1^{\text{th}}$ stage contains a derivation $d$ of $\Gamma, \varphi \vdash \varphi \to \psi$.

    We must show $\Gamma, \varphi \models \varphi \to \psi$. 
    
    Since $d_{\psi} \in Stage_{Der}(n)$ we have $\Gamma, \varphi \models \psi$.

    Let $\MODEL$ be a model which $\MODEL \models \gamma$ for all $\gamma \in \Gamma \cup \varphi$.

    Thus we have $\MODEL \models \varphi$ and $\MODEL \models \psi$, so $\MODEL \models \varphi \to \psi$.

    \item Suppose:
        \begin{itemize}
            \item $d_{\exists}$ is a derivation of $\Gamma \vdash \exists x \varphi (x)$
            \item $d_{\psi}$ is a derivation of $\Delta, \varphi(a) \vdash \psi$, where $a$ does not occur in $\Delta$ or $\exists x \varphi (x)$
        \end{itemize}

        where all are members of stage $n$. Then the $(\exists E)$ rule tells us that the $n + 1^{\text{th}}$ stage contains a derivation $d$ of $\Gamma, \Delta, \varphi (a) \vdash \psi$.

        We must show that $\Gamma, \Delta, \varphi (a) \models \psi$.

        Since $d_{\exists} \in Stage_{Der}(n)$, we have $\Gamma\models \exists x \varphi (x)$.

        Since $d_{\psi} \in Stage_{Der}(n)$, we have $\Delta, \varphi (a) \models \psi$.


    Let $\MODEL$ be a model which $\MODEL \models \gamma$ for all $\gamma \in \Delta \cup \varphi (a)$.
        Then $\MODEL \models \psi$.


        Suppose that $m = a^{\MODEL}$. Now consider any model $\MODEL'$ which is exactly like $\MODEL$ except that $a^{\MODEL'} = m' \neq a^{\MODEL} = m$; i.e., we let $a^{\MODEL'}$ be some arbitrary other $m'$ from the domain $M$. Since $\Delta \cup \varphi (a)$ has a sentence with the constant symbol $a$ in it, $\MODEL'$ only models $\psi$ if it models  $\varphi (a)$.


        But the fact that all models $\MODEL'$ whose only difference from $\MODEL$ is their interpretation of the symbol $a$ are such that $\MODEL' \models \psi$ just means that:

        $\exists m \in M\ \MODEL' \models \exists x \varphi (x), \varphi (m)\Leftrightarrow \MODEL \models \psi$.

\end{enumerate}


\end{document}