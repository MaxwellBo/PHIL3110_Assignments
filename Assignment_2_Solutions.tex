\documentclass[a4paper]{article}
\usepackage{fullpage}
\usepackage{amsmath}
\usepackage{amssymb}
\usepackage{bussproofs}
\usepackage{verbatim}

\newtheorem{lemma}{Lemma}
\newtheorem{theorem}{Theorem}
\newtheorem{definition}{Definition}

\newtheorem{proof}{Proof}

\newcommand{\MODEL}{\mathcal{M}}
\newcommand{\LANGUAGE}{\mathcal{L}}
\newcommand{\TUPLE}[1]{\langle {#1} \rangle}
\newcommand{\SET}[1]{\{ {#1} \}}
\newcommand{\PV}{\varphi}
\newcommand{\QV}{\psi}

\title{PHIL3110 - Assignment 2}
\author{Maxwell Bo}

\begin{document} 

\maketitle

\subsection*{Problem 1}

\begin{definition}
$f: \mathcal{X} \rightarrowtail \mathcal{Y}$ is is injective $\Leftrightarrow$ $\forall x_1, x_2 \in \mathcal{X}$ if $F(x_1) = F(x_2)$ then $x_1 = x_2$.
\end{definition}

\begin{theorem}
    If $f: \mathcal{X} \rightarrowtail \mathcal{Y}$ is injective and $g: \mathcal{Y} \rightarrowtail \mathcal{Z}$ is injective, then $g \circ f$ is injective.
\end{theorem}

\begin{proof}
Suppose $f: \mathcal{X} \rightarrowtail \mathcal{Y}$ is injective and $g: \mathcal{Y} \rightarrowtail \mathcal{Z}$ is injective. We must show that $g \circ f$ is injective. Suppose $x_1$ and $x_2$ are elements of $\mathcal{X}$ such that

\[(g \circ f)(x_1) = (g \circ f)(x_2)\]

By definition of composition of functions,

\[g(f(x_1)) = g(f(x_2))\]

Since $g$ is injective

\[f(x_1) = f(x_2)\]

And since $f$ is injective

\[x_1 = x_2\]

\end{proof}

\begin{theorem}
If there is some injection $f: A \rightarrowtail B$, then $A \precsim B$, and vice-versa.
\end{theorem}

If $A$, $B$ and $C$ are sets such that $A \precsim B$ and $B \precsim C$, there exists an injective $f: A \rightarrowtail B$ and injective $g: B \rightarrowtail C$. Per theorem 1, $g \circ f: A \rightarrowtail C$ is injective. Per thereom 2, $g \circ f$ implies $A \precsim C$.

\subsection*{Problem 2}

\begin{enumerate}

    \item

\[\varphi_2 := (\exists x)(\exists y) (x \neq y)\] says that there at least 2 objects.

Thus \[\varphi_n := (\exists x_1)(\exists n_2) \ldots (\exists x_n)( x_1 \neq \ x_2 \neq \ldots \neq x_n) \] says that for any natural number $n$, there are at least $n$ objects.

    \item


\end{enumerate}


\newpage
\subsection*{Problem 3}

\verbatiminput{PHIL3110Turing.txt}



\end{document}