\documentclass[a4paper]{article}
\usepackage{fullpage}
\usepackage{amsmath}
\usepackage{amssymb}
\usepackage{bussproofs}
\usepackage{verbatim}

\newtheorem{lemma}{Lemma}
\newtheorem{theorem}{Theorem}
\newtheorem{definition}{Definition}

\newtheorem{proof}{Proof}

\newcommand{\MODEL}{\mathcal{M}}
\newcommand{\LANGUAGE}{\mathcal{L}}
\newcommand{\TUPLE}[1]{\langle {#1} \rangle}
\newcommand{\SET}[1]{\{ {#1} \}}
\newcommand{\PV}{\varphi}
\newcommand{\QV}{\psi}

\title{PHIL3110 - Assignment 2}
\author{Maxwell Bo}

\begin{document} 

\maketitle

\subsection*{Problem 1}

\begin{definition}
$f: \mathcal{X} \rightarrowtail \mathcal{Y}$ is injective $\Leftrightarrow$ $\forall x_1, x_2 \in \mathcal{X}$ if $F(x_1) = F(x_2)$ then $x_1 = x_2$.
\end{definition}

\begin{theorem}
    If $f: \mathcal{X} \rightarrowtail \mathcal{Y}$ is injective and $g: \mathcal{Y} \rightarrowtail \mathcal{Z}$ is injective, then $g \circ f$ is injective.
\end{theorem}

\begin{proof}
Suppose $f: \mathcal{X} \rightarrowtail \mathcal{Y}$ is injective and $g: \mathcal{Y} \rightarrowtail \mathcal{Z}$ is injective. We must show that $g \circ f$ is injective. Suppose $x_1$ and $x_2$ are elements of $\mathcal{X}$ such that

\[(g \circ f)(x_1) = (g \circ f)(x_2)\]

By definition of composition of functions,

\[g(f(x_1)) = g(f(x_2))\]

Since $g$ is injective

\[f(x_1) = f(x_2)\]

And since $f$ is injective

\[x_1 = x_2\]

\end{proof}

\begin{theorem}
There is some injection $f: A \rightarrowtail B$ $\Leftrightarrow$ $A \precsim B$.
\end{theorem}

If $A$, $B$ and $C$ are sets such that $A \precsim B$ and $B \precsim C$, there exists an injective $f: A \rightarrowtail B$ and injective $g: B \rightarrowtail C$. Per theorem 1, $g \circ f: A \rightarrowtail C$ is injective. Per thereom 2, $g \circ f$ entails $A \precsim C$.

\subsection*{Problem 2}

\begin{enumerate}

    \item

\[\varphi_1 := (\exists x)(x)\] says that there is at least 1 object, and

\[\varphi_2 := (\exists x)(\exists y) (x \neq y)\] says that there at least 2 objects.

Thus \[\varphi_n := (\exists x_1)(\exists x_2) \ldots (\exists x_n)( x_1 \neq x_2 \neq \ldots \neq x_n) \] says that for any natural number $n$, there are at least $n$ objects.

    \item

    We need to prove that every finite subset $\Delta$ of $S$ has a model, e.g. there exists a model $\mathcal{M}$ such that $\mathcal{M} \models \Delta$.

    First, we'll define a stage based defintion of $S$, where:

    \begin{align*}
        s(0) & = T\\
        s(n + 1) & = s(n) \cap \SET { \varphi_{n + 1}}
    \end{align*}

    such that $\SET{ s(n) \mid n \in \omega } = S$.

    At each step, we need to show that

    \begin{itemize}
        \item The model that satisfies $s(n)$ defines $s(n)$ objects
        \item The model is finitely satisfiable 
        
    \end{itemize}

    \subsubsection*{Base} 
    $0 \in A$. We can fix some $\mathcal{M}$ where $\mathcal{T} \subseteq \mathcal{M}$, where $\mathcal{T} \models T$ for every finite subset $\Delta$ of $s(0)$.
    
        The only finite subset of $s(0)$ is $T$.
        
        By our fix of $\mathcal{M}$, $\mathcal{M} \models T$. Therefore, $s(0)$ is finitely satisfiable - $\mathcal{M} \models s(0)$.
    
        $\mathcal{M}$ defines at least 0 objects.

    \subsubsection*{Induction Step} 

    Suppose $n \in A$. 
    
    Let $\mathcal{N}$ be the model that satisfied $s(n)$. It defined at least $n$ objects. 

    $\mathcal{N} \not\models \Delta$ for each finite subset $\Delta$ of $s(n + 1)$.

    Why?

    The only finite subset $\Delta$ of $s(n + 1) \setminus s(n)$ is $\SET{\varphi_{n + 1}}$.

    Consider the only sentence in $\Delta$, $\varphi_{n + 1}$. 

    In order for $\mathcal{N} \models \varphi_{n + 1}$, $\mathcal{N}$ would need to define $n + 1$ objects.

    Consider some $\mathcal{M}$ that defines a new object, so that $\mathcal{N} \subseteq \mathcal{M}$. Now $\mathcal{M}$ defines at least $n + 1$ objects. 

    By theorem 139, as $\gamma$ is $\Sigma_1$, and $\mathcal{N} \models \varphi_n$\footnote{$\varphi_n \in s(n) \setminus s(n -1)$}, $M \models \varphi_{n + 1}$.
    
    As $\mathcal{M} \models s(n)$, and $\mathcal{M} \models s(n + 1) \setminus s(n)$, every finite subset $\Delta$ of $s(n) \cup (s(n + 1) \setminus s(n))$ is satisfiable. Therefore $s(n + 1)$ is finitely satisfiable - $\mathcal{M} \models s(n + 1)$ 

    Thus, $n + 1 \in A$. Then by induction, we see that every $n \in A$.

    Every finite subset $\Delta$ of $S$ has a model.

    \item 

\end{enumerate}

\newpage
\subsection*{Problem 3}

\verbatiminput{PHIL3110Turing.txt}

\subsection*{Problem 4}

We will attempt to demonstrate that $A$ is recursive. The same process may be repeated to show that $B$ and $C$ are also recurisve.

By definition, $A$ is recursively enumerable, and that $A \subseteq \omega$. To prove that $A$ is recurisive, we must show that  $\omega \setminus A$ is recursively enumerable.

    Suppose $\omega \setminus A$ was not recursively enumerable.

    There exists an $n \in \omega \setminus A$, such that no $\varphi_e$ exists, that has a domain $\omega \setminus A$, and halts when applied to $n$.

    As $B$ and $C$ are recursively enumerable there are partial recursive functions $\varphi_B$ and $\varphi_C$ which have the domains of $B$ and $C$ respectively. We can define a function 
    $$
    \varphi_{B \cup C}(n) = 
    \begin{cases}
        \varphi_b(n), &\mbox{if } \varphi_b(n) \text{ halts.}\\
        \varphi_c(n), &\mbox{if } \varphi_c(n) \text{ halts.}\\
        \text{undefined,} &\mbox{ otherwise.} 
    \end{cases} 
    $$

    which has the domain $B \cup C$. By definition $A \cap B \cap C = \omega$, $B \cap C = \omega \setminus A$.

    This means we have found a partially recursive function, $\varphi_{B \cup C}$, that has the domain $n \in \omega \setminus A$ and halts on every $n \in \omega \setminus A$. This is a contradiction.

    Thus, $\omega \setminus A$ is recursively enumerable.

    Thus, $A$ is recursive.
    

\end{document}